% !TeX spellcheck = pt_BR
\chapter{Marketing}

\textbf{Marketing}, pela definição da Associação Americana de Marketing, é o planejamento e a execução de atividades de concepção definição de preço, promoção e distribuição de ideias, produtos e serviços para criar valores que satisfaçam a objetivos individuais e organizacionais \cite{maximiano}. Pode ser resumido como determinar as necessidades e os desejos dos clientes (mercado-alvo) e satisfaze-los (oferecendo um produto) de forma mais eficaz que os concorrentes \cite{maximiano}.


Um \textbf{produto} pode ser  um objeto físico, bens, serviços, pessoas, lugares, organizações, atividades e ideias \cite{maximiano}.


\section{Mercado}

\textbf{Mercado} é um grupo de consumidores que têm necessidade e interesses similares com poder aquisitivo e disposição para comprar \cite{maximiano}. \textbf{Mercado consumidor}  são consumidores que compram o produto para seu próprio uso. \textbf{Mercado industrial} são consumidores que compram os produtos para revende-los ou usá-los em sua operações \cite{maximiano}.


O mercado pode ser dividido em \textbf{população em geral} (toda a sociedade), \textbf{mercado potencial} (pessoas que se interessam por um produto), \textbf{mercado-alvo} (pessoas que a organização pretende conquistar) e \textbf{mercado conquistado} (consumidores que compram os produtos da organização). Entender, definir e estudar as pessoas que participam de cada grupo é a base para planejar o investimento nas atividades de marketing \cite{maximiano}.


Os clientes podem ser \textbf{clientes que usam e pagam} (usam diretamente o produto), \textbf{clientes que usam e pagam} (por exemplo, crianças que usam o que os pais compram), \textbf{clientes que não usam e pagam} (compram para outro utilizarem) \cite{maximiano}.


O \textbf{processo de decisão de compra} apresenta as etapas: necessidade, busca de informação, avaliação de alternativas, compra e resultados \cite{maximiano}.


Os fatores que influenciam po processo de compra dos consumidores são a cultura (conjunto de valores que regem como as pessoas agem), classe social (poder aquisitivo), grupos de consumidores (consumidores podem ser agrupados por sua preferências), auto-imagem (algumas pessoas pensam que "você é aquilo que compra/usa"), fator situacional (feriados, aniversários, etc.), tipo de mercado (vender para consumidores finais é diferente do que vender para o mercado industrial) \cite{maximiano}.


A \textbf{segmentação do mercado} é a divisam do mercado em grupos relativamente homogênicos, com necessidades semelhantes. Cada segmento poder ser alvo de um produto desenvolvido especialmente parar ele (diferenciação do produto) \cite{maximiano}.


As etapas no processo de planejamento estratégico de marketing são segmentação do mercado, posteriormente, escolha do marcado-alvo e, por fim, a definição do marketing \cite{maximiano}.


Os critérios de segmentação do mercado consumidor são demografia (idade, sexo, renda, etnia e ocupação), geografia (localização) e psicologia (motivações pessoais) \cite{maximiano}.


Os critérios de segmentação do mercado industrial são tipo de cliente (tipo de negócio do cliente), volume de negócios (quantidade de compra), localização, aplicação do produto (o que fazem do produto comprado) \cite{maximiano}.


A identificação do mercado-alvo deve avaliar o tamanho e crescimento de mercado, atratividade, objetivo e recursos da empresa.


O \textbf{marketing mix} é o conjunto singular de decisões de cada empresa a respeito dos quatro Ps (produto, preço, praça/logística e promoção). O objetivo das decisões é obter uma resposta positiva dos clientes \cite{maximiano}.


\section{Produto}

A diferenciação do produto é percebida pelo seu design, sua embalagem e marca \cite{maximiano}.


A \textbf{qualidade} do produto é avaliado pelo seu desempenho, acessórios, confiabilidade, durabilidade, facilidade de manutenção, estética (depende de preferências pessoais), qualidade percebida (julgamentos subjetivos do produto) e excelência (o melhor produto que se pode fazer) \cite{maximiano}.


O ciclo de vida do produto apresenta quatro fases: introdutória (clientes são informados do produto através promoções e preços estratégicos), crescimento (vendas apresentam crescimento rápido),maturidade (grande competitividade, devendo ser lidada com marketing agressivo e diferenciação do produto ou estratégia de preços), declínio (vendas caem) \cite{maximiano}.


\section{Preço}

Os fatores que afetem a decisão do preço são os objetivos de marketing da empresa, regulamentação governamental e percepções dos consumidores (preço baixo gera baixa qualidade do produto, preços elevados gera questionamento se realmente vale este valor) \cite{maximiano}.


A determinação do preço deve levar em consideração o custo de produção, preço premium (produto com vantagens competitivas pode ser vendido a preços acima da media do mercado), preço de penetração (preços baixos para entrar no mercado, conquistando consumidores e, posteriormente, o preço é aumentado), preço de desnatação (laça o produto com preço alto, aproveitando o ineditismo, e diminui para manter as vendas estáveis), preço de defesa (diminui o preço quando as vendas caem, ou se o concorrente reduz o preço) \cite{maximiano}.

\section{Distribuição}


Os \textbf{nanais de distribuição}  são os meios que levam os produtos até os consumidores. A presença de  agentes intermediários, varejistas, por exemplo, reduz o tempo e custo do processo de compra e venda entre consumidores finais e produtores \cite{maximiano}.


Há quatro formar principais de distribuição: direta (não há intermediários), indireta (há intermediários, podendo ser varejista, atacadista e varejista, vendedor ou corretor, atacadista e varejista) \cite{maximiano}.


Os canais de distribuição apresentam as funções de fracionar (dividir adequadamente grandes quantidades produzidas entre os pedidos de cada consumidor), direcionar ao publico-alvo (agrupar produtos de um mesmo segmento de consumo), fornecer informações (sobre o mercado ao produtor) e transferir riscos (do negócio aos intermediários) \cite{maximiano}.


A escolha do canal de distribuição depende da cobertura do mercado (de forma apropriada ao produto), custo (o produto deve estar disponíveis a um grande número de consumidores e regiões geográficas, mas eleva o custo), controlo (de como onde e quando o produto é vendido) \cite{maximiano}.

\section{Promoção}


\textbf{Promoção} é a ação de informar ou lembrar consumidores sobre um produto ou marca, por meio de várias técnicas de comunicação e persuasão \cite{maximiano}.


As quatro formas mais usada pelas empresas para fazer a promoção são a propaganda, representantes de vendas (pessoa que influencia diretamente os compradores), ofertas promocionais (descontos, cupons, amostras grátis e brindes) \cite{maximiano}.


As \textbf{relações publicas} são atividades que têm como objetivo criar/manter uma imagem pública favorável, transmitindo uma boa imagem para os produtos da empresa. As atividades mais comuns são eventos especiais, informativos oficiais (usado para esclarecer informações sobre a empresa), conferência de imprensa (reunião que informar sobre ações a serem tomadas pela empresa) \cite{maximiano}.