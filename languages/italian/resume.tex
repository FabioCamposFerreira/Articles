% !TeX spellcheck = it_IT
\section{Alfabeto}
	\begin{tabular}{cc}
Letra&Pronunciación\\
	A& a \\
	B&bi\\
	C&chi\\
	D&di\\
	E&e\\
	F&effe\\
	G&gi\\
	H&acca\\
	I&i\\
	L&elle\\
	M&emme\\
	N&enne\\
	O&o\\
	P&pi\\
	Q&ku\\
	R&erre\\
	S&esse\\
	T&ti\\
	U&u\\
	V&vi\\
	Z&zeta\\
SCI	&schi\\
	SCE	&sche\\
		GN	&ñ\\
			GL	&ll diante de i\\
\end{tabular}
\section{Articolo}
\subsection{Articolo indeterminativo}
\begin{tabular}{cc}
\multicolumn{2}{x}{Maschile}\\
un& caso generale\\
uno (davanti a parole che iniziano per \textbf{z}, \textbf{gn}, \textbf{x}, \textbf{ps}, \textbf{s} impura, \textbf{i} semiconsonante e, variabilmente, pn[1])\\
\multicolumn{2}{x}{Femminile}\\
una& caso generale\\
un' & (davanti a parole che iniziano per vocale, ma non davanti a \textbf{i} semiconsonantica)
\end{tabular}
\subsection{Articolo determinativo}
\begin{tabular}{cc}
\multicolumn{2}{c}{Maschile}\\
\multicolumn{2}{c}{Singolare}\\
il& caso generale\\
lo& davanti a parole che iniziano per z, gn, x, ps, s impura, i semiconsonante e, variabilmente, pn; eliso in l' davanti a parole che iniziano per vocale)\\
\multicolumn{2}{c}{Plurale}\\
i& caso generale\\
gli& davanti a parole che iniziano per z,x, gn, ps, s impura o vocale)\\
\multicolumn{2}{c}{Femminile}\\
\multicolumn{2}{c}{Singolare}\\
la& caso generale\\
\multicolumn{2}{c}{Plurale}\\
le& caso generale\\
\end{tabular}
\section{Subtantivos}
\begin{tabular}{cc}
\multicolumn{2}{c}{Maschile}\\
Singolare&Plurale\\
\textbf{-o}&\textbf{-i}\\
\textbf{-e}&\textbf{-i}\\
\multicolumn{2}{c}{vocale accentata}\\
\multicolumn{2}{c}{consonante finale}\\
\multicolumn{2}{c}{\textbf{-ma}}\\
\multicolumn{2}{c}{\textbf{-a}}\\
\multicolumn{2}{c}{Femminile}\\
Singolare&Plurale\\
\textbf{-o}&\textbf{-e}\\
\textbf{-e}&\textbf{-i}\\
\multicolumn{2}{c}{vocale accentata}\\
\multicolumn{2}{c}{\textbf{-o}}\\
\multicolumn{2}{c}{\textbf{-si} (parole di origine greca)}\\
\end{tabular}
\section{Aggettivo}
\subsection{Aggettivi qualificativi}
\begin{tabular}{cc}
	\multicolumn{2}{c}{Maschile}\\
	Singolare&Plurale\\
	\textbf{-o}&\textbf{-i}\\
	\textbf{-e}&\textbf{-i}\\
	\multicolumn{2}{c}{Femminile}\\
	Singolare&Plurale\\
	\textbf{-o}&\textbf{-e}\\
	\textbf{-e}&\textbf{-i}\\
\end{tabular}
\subsection{Aggettivi possessivi}
\begin{tabular}{ccccc}
&\multicolumn{2}{c}{Maschile}&\multicolumn{2}{c}{Femminile}\\
&Singulare&Plurale&Singulare&Plurale\\
1\textordfeminine persona singolare&mio&miei&mia&mie\\
2\textordfeminine persona singolare&tuo&tuoi&tua&tue\\
3\textordfeminine persona singolare&suo,proprio&suoi,propri&sua,propria&sue,proprie\\
1\textordfeminine persona plurale  &nostro&nostri&nostra&vostre\\
2\textordfeminine persona plurale  &vostro&vostri&vostra&vostre\\
3\textordfeminine persona plurale  &loro, proprio&loro, propri&loro, propria&loro, proprie\\
\end{tabular}
\subsection{Graus de comparação}
\subsection{Adjetivos possesivos}
\subsection{Adjetivos demonstrativos}
\section{Pronome}
\section{Pronomi Personali}
\begin{tabular}{ccccccccc}
&nominativo&genitivo/ablativo&dativo&accusativo/tonico/preposizionale&accusativo atono& combinazione&strumentale/locativo&lativo\\
1\textordfeminine persona singolare&io&-&mi&me&mi&me lo& -& -\\
2\textordfeminine persona singolare&tu&-&ti&te&ti&te lo&-&-\\
3\textordfeminine persona singolare maschile&egli,esso&ne&gli,si&lui,sé&lo, si&glielo, se lo&ci&vi/ci\\
3\textordfeminine persona singolare femminile&ella,essa&ne&le,si&lei,sé&la, si& gliele, se lo&ci&vi/ci\\
1\textordfeminine persona plurale  &noi&-&ci&noi&ci&ce lo&-&-\\
2\textordfeminine persona plurale  &voi&-&vi&voi&vi&ve lo&-&-\\
3\textordfeminine persona plurale  maschile&essi&ne&loro, si&loro, sé&li, si& se lo& ci& vi/ci\\
3\textordfeminine persona plurale  femminile&esse&ne&loro, si&loro, sé&le, si&se lo& ci& vi/ci\\
\end{tabular}
\section{Numeros}
\section{Verbos}
\subsection{Presente del indicativo}
Presente del indicativo  é utilizado para falar de ações habituais e ações futuras.


Usando o presente de lindicativo no afirmativo: Noi lavoriamo in um ufficio.


Usando o presente de lindicativo no negativo, adiciona-se \textbf{non} entre o sujeto e o verbo: Lei non dorme.


Usando o presente de lindicativo no negativo, não muda: Lavori in in ufficio?


Verbos terminados em \textbf{-are} no indicativo, segue a tabela de flexão tendo como exemplo \textbf{mangiare}
\begin{tabular}{cc}
Io&mang\textbf{io}\\
Tu&mang\textbf{i}\\
Lui/Lei&mang\textbf{ia}\\
Noi&mang\textbf{iamo}\\
Voi&\textit{mang\textbf{iate}}\\
Loro&mang\textbf{iano}\\
\end{tabular}


Verbos terminados em \textbf{-ere} no indicativo, segue a tabela de flexão tendo como exemplo \textbf{leggere}
\begin{tabular}{cc}
	Io&legg\textbf{o}\\
	Tu&legg\textbf{i}\\
	Lui/Lei&legg\textbf{e}\\
	Noi&legg\textbf{iamo}\\
	Voi&\textit{legg\textbf{ete}}\\
	Loro&legg\textbf{ono}\\
\end{tabular}


Verbos terminados em \textbf{-ire} no indicativo, segue a tabela de flexão tendo como exemplo \textbf{dormire}
\begin{tabular}{cc}
	Io&dorm\textbf{o}\\
	Tu&dorm\textbf{i}\\
	Lui/Lei&dorm\textbf{e}\\
	Noi&dorm\textbf{iamo}\\
	Voi&\textit{dorm\textbf{ite}}\\
	Loro&dorm\textbf{ono}\\
\end{tabular}

\section{Adverbios}
\section{Preposizione}
\subsection{Preposizioni Proprie}
\begin{tabular}{c}
di\\
a\\
da\\
in\\
con\\
su\\
per\\
tra\\
fra
\end{tabular}
 Le preposizioni proprie semplici possono essere unite agli articoli determinativi e formare così le preposizioni articolate.
\begin{tabular}{cc}
del&di+il\\
dello&di+lo\\
della&di+la\\
dei&di+i\\
degli&di+gli\\
delle&di+le\\
al&a+il\\
allo&a+lo\\
alla&a+la\\
ai&a+i\\
agli&a+gli\\
alle&a+le\\
dal&da+il\\
dallo&da+lo\\
dalla&da+la\\
dai&da+i\\
dagli&da+gli\\
dalle&da+le\\
nel&in+il\\
nello&in+lo\\
nella&in+la\\
nei&in+i\\
negli&in+gli\\
nelle&in+le\\
col&con+il\\
collo&con+lo\\
colla&con+la\\
coi&con+i\\
cogli&con+gli\\
colle&con+le\\
sul&su+il\\
sullo&su+lo\\
sulla&su+la\\
sui&su+i\\
sugli&su+gli\\
sulle&su+le\\
\end{tabular}
\subsection{Preposizioni Improprie}
\begin{tabular}{c}
sopra\\
sotto\\
prima\\
dopo\\
vicino\\
$\cdots$
\end{tabular}
