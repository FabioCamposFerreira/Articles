% !TeX spellcheck = pt_BR
\chapter{Python}

O Python é uma linguagem de programação interpretada, interativa e orientada a objetos. Incorpora módulos, exceções, tipagem dinâmica e forte, tipo de dados  dinâmicos de alto nível e classes. Suporta multipolos paradigmas de programação como programação orientada a objetos, programação funcional e procedural. Possui uma sintaxe limpa \cite{pythonDocs}.

\chapter{How to run Python}
\section{Windows}
\section{Linux}
\chapter{Structure}
\chapter{Data Types}
\chapter{if}
\chapter{Switch}
\chapter{While}
\chapter{For}
\chapter{Function}
\chapter{Class}
\chapter{Frequent codes}
\chapter{Principais funções}
\section{Como pegar o texto digitado pelo terminal?}
\begin{lstlisting}[language=python]
	name = input("What is yout name? ")
	# What is yout name? Fabio
  print(name)
  # Fabio
  \end{lstlisting}
  \section{Como imprir textos coloridos?}
\begin{lstlisting}[language=python]
	text = "Text red"
	print("\033[91m {}\033[00m".format(text))
	# Text red
	text = "Text green"
	print("\033[92m {}\033[00m".format(text))
	# Text green
  \end{lstlisting}
\chapter{Exercises/Examples}
